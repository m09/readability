%%% Local Variables: 
%%% coding: utf-8
%%% mode: latex
%%% TeX-engine: xetex
%%% End: 
\documentclass[a4paper]{scrreprt}

\usepackage{fontspec}
\setmainfont{Linux Libertine O}
\usepackage{polyglossia}
\usepackage{amsmath}
\usepackage{xcolor}
\usepackage{graphicx}
\usepackage{listings}
\usepackage{caption}
\usepackage{wrapfig}
\usepackage{float}

\renewcommand\chaptername{Section}
\renewcommand\thechapter{\Roman{chapter}}

\DeclareGraphicsExtensions{.png, .jpeg, .jpg, .svg, .eps}

\begin{document}
\begin{titlepage}
  \begin{center}
    \noindent\rule{\textwidth}{0.4pt}
    {\huge\bfseries Improving text readability using Simple
      Wikipedia\\}
    \noindent\rule{\textwidth}{0.4pt}\\
    % ----------------------------------------------------------------
    \vspace{1.5cm}
    {\large
      \textbf{Author:} \textsc{Hugo Mougard}
      \hfill
      \textbf{Advisor:} \textsc{Akiko Aizawa}}\\[5pt]
    % ----------------------------------------------------------------
    \vfill
    \parbox{0pt}{\large \begin{tabbing}
        \textbf{Sending institution:} \= National Institute of Informatics \kill
        \textbf{Sending institution:} \> Université de Nantes, France \\
        \textbf{Hosting institution:} \> National Institute of
        Informatics, Japan \\
      \end{tabbing}}
    \vfill
    {\LARGE ATAL Master 2 internship report}
    \vfill
    % ----------------------------------------------------------------
    \includegraphics[width=0.19\textwidth]{img/nii.png}
    \hfill
    \includegraphics[width=0.19\textwidth]{img/lina.png}
    \hfill
    \includegraphics[width=0.19\textwidth]{img/univ.eps}
    \vfill
    {July 2014}
  \end{center}
\end{titlepage}

\tableofcontents
\chapter{Introduction, description de la problématique}

\chapter{Présentation de l'entreprise ou du laboratoire}

\section{National Institute of Informatics}
\label{sec:national-institute-of-informatics}

My internship takes place at the National Institute of Informatics,
Tokyo, Japan. 

\section{Aizawa Laboratory}
\label{sec:aizawa-laboratory}

The team that welcomed me for this work is Aizawa Laboratory. It is
led by Professor Akiko Aizawa. At the time of this writing, the
laboratory has 18 members and strong partnerships with ex-members of
the laboratory.

Its expertise lies in several sub-domains of NLP: gaze-NLP, Analysis
and mining of scientific papers, mathematical information retrieval, etc…

\chapter{Description à la fois globale (l'intérêt du sujet) et précise (quelle est la question) du sujet}

\chapter{State of the art}
\label{sec:sota}

Readability as been studied extensively. The first works in this area
date back to almost a century ago: Thorndike computed list of ten
thousand easy words. This list got used shortly thereafter by teachers
trying to select good books for children learning to read went through
the process of estimating the readability of many books.

More recently, from 2006 onwards, Simple English Wikipedia has been
rightly seen as an important resource for readability studies. Many
scientists have used it to compute parallel corpora of readable their
hard-to-read counterparts sentences.

\chapter{Etude de l'existant (solutions techniques)}



\chapter{Descriptif de la solution (selon le sujet, analyse algorithmique, technique, technologique, etc)}
\chapter{Conclusions  et perspectives}


\end{document}
